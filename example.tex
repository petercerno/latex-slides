\documentclass[10pt,varwidth=6in,margin=0.2in,preview]{standalone}
\usepackage[english]{babel}
\usepackage[utf8]{inputenc}
\usepackage{amssymb}
\usepackage{amsmath}
\usepackage{enumitem}
\usepackage{adjustbox}
\usepackage[dvipsnames]{xcolor}

\definecolor{pagecolor}{rgb}{1,0.98,0.9}

\title{
    Principles of Mathematical Analysis (3rd Edition)\\
    by Walter Rudin
}
\author{
    Peter Cerno\\
    \small petercerno[at]gmail.com
}

\begin{document}

\pagecolor{pagecolor}

% __BEGIN

\date{}
\maketitle

\begin{abstract}
These are my solutions to exercises from the book \emph{Principles of Mathematical Analysis} by \emph{Walter Rudin} (3rd edition).
\end{abstract}

% __SLIDE

\setcounter{section}{0}
\section{The Real and Complex Number Systems}

\setcounter{subsection}{5}
\subsection{Exercise}

Fix $b > 1$.

% __PAUSE

\begin{enumerate}[label=(\alph*)]

\item
If $m, n, p, q$ are integers, $n > 0, q > 0$, and $r = \frac{m}{n} = \frac{p}{q}$, prove that
\[ (b^m)^{1/n} = (b^p)^{1/q} \]
Hence it makes sense to define $b^r = (b^m)^{1/n}$.

% __ADD::\end{enumerate}
% __PAUSE

\item
Prove that $b^{r+s} = b^r b^s$ if $r$ and $s$ are rational.

\end{enumerate}

% __SLIDE

\textbf{Proof}.

\vspace{0.1in}

(a) If $m, n, p, q$ are integers, $n > 0, q > 0$, and $r = \frac{m}{n} = \frac{p}{q}$, prove that $(b^m)^{1/n} = (b^p)^{1/q}$.
% __PAUSE

\vspace{0.1in}

$\frac{m}{n} = \frac{p}{q}$ implies $mq = pn$, thus $b^{mq} = b^{pn} > 0$, i.e $(b^{mq})^{1/nq} = (b^{pn})^{1/nq}$.
% __PAUSE
Let us show that $(b^{mq})^{1/nq} = (b^m)^{1/n}$ and $(b^{pn})^{1/nq} = (b^p)^{1/q}$.
% __PAUSE
Apparently, $\left( (b^m)^{1/n} \right)^{nq} = \left[ \left( (b^m)^{1/n} \right)^n \right]^q = (b^m)^q = b^{mq}$, which implies that $(b^{mq})^{1/nq} = (b^m)^{1/n}$.
% __PAUSE
Similarly, $(b^{pn})^{1/nq} = (b^p)^{1/q}$.
% __PAUSE
Therefore, $(b^m)^{1/n} = (b^{mq})^{1/nq} = (b^{pn})^{1/nq} = (b^p)^{1/q}$. $\square$
% __PAUSE

\vspace{0.1in}

(b) Prove that $b^{r+s} = b^r b^s$ if $r$ and $s$ are rational.
% __PAUSE

\vspace{0.1in}

Let us assume that $r = \frac{p}{n}$ and $s = \frac{q}{n}$ for some positive integer $n$.
% __PAUSE
According to (a): $b^{r+s} = b^{(p+q)/n} = (b^{p+q})^{1/n}$, thus $(b^{r+s})^n = b^{p+q}$.
% __PAUSE
On the other hand, $(b^r b^s)^n = (b^r)^n (b^s)^n = \left( (b^p)^{1/n} \right)^n \left( (b^q)^{1/n} \right)^n = b^p b^q = b^{p+q}$.
% __PAUSE
Together we have $(b^{r+s})^n = b^{p+q} = (b^r b^s)^n$, which implies (b). $\square$

% __END

\vspace{4in}

\hrule

\end{document}
