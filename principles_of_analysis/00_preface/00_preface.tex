\documentclass[10pt,varwidth=6in,margin=0.2in,preview]{standalone}
\usepackage[document]{ragged2e}
\usepackage[english]{babel}
\usepackage[utf8]{inputenc}
\usepackage{amssymb}
\usepackage{amsmath}
\usepackage{enumitem}
\usepackage{adjustbox}
\usepackage[dvipsnames]{xcolor}

\definecolor{pagecolor}{rgb}{0.02,0.07,0.21}
\definecolor{textcolor}{rgb}{0.78,0.77,0.44}

\title{
    Principles of Mathematical Analysis (3rd Edition)\\
    by Walter Rudin
}
\author{
    Series of Lectures by Peter Cerno\\
    \small petercerno[at]gmail.com
}

\begin{document}

\pagecolor{pagecolor}
\color{textcolor}

% __BEGIN

\date{}
\maketitle

\begin{abstract}
This series of lectures on mathematical analysis is based on Walter Rudin's seminal work: \emph{Principles of Mathematical Analysis} (3rd Edition). While useful as a supplementary resource, the original book still remains an indispensable reference.
\end{abstract}

% __NOTE: Welcome to a series of lectures meticulously crafted to complement the book Principles of Mathematical Analysis, Third Edition, by Walter Rudin, an esteemed text in the field. This supplementary material is thoughtfully designed to enhance and elevate your exploration through the complex, yet fascinating world of mathematical analysis.
% __NOTE: Embracing the original book's established structure, we adhere to its chapter and section numbering to ensure a seamless and integrated learning experience. Yet, our slides delve deeper into the proofs, furnishing them with enriched detail and supplementary commentary that lightens the path of understanding. We have taken the initiative to provide elaborate discussions, stimulating critical thought and offering new perspectives on fundamental concepts. Moreover, we extend a helping hand with solutions and substantial hints for exercises, offering a robust framework to support and empower your learning journey.
% __NOTE: Our material is tailored for advanced undergraduates and those delving deep into mathematics. As for prerequisites, we expect a foundational understanding of integers, rational numbers, and arithmetic. A basic grasp of standard mathematical notations, deductive reasoning, logic, as well as the concepts of sets and their operations, are assumed as well.
% __NOTE: Our main motivation and driving force in developing this project is the creation of an open-source lecture series on mathematical analysis that not only elucidates the subject matter, but also ensures it is comprehensible and accessible to a wide audience. Embark on this enlightening journey with us and delve into the profound depths of mathematical analysis, made clearer and more engaging.

% __END

\vspace{4in}

\hrule

\end{document}
